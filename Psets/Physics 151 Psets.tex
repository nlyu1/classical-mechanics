\documentclass[12pt]{article}

% Set margins
\usepackage[margin=0.5in]{geometry}
% Remove paragraph indentation
\setlength{\parindent}{1\baselineskip}
\setlength{\parskip}{0.5\baselineskip}

\usepackage{amsmath}
\usepackage{amssymb}
\usepackage{amsfonts}
\usepackage{hyperref}
\usepackage{amsthm}
\usepackage{tikz-cd}
\usepackage{tikz}
\usepackage{cancel}
\usepackage{mdframed}
\usepackage{enumitem}
\usepackage{tikz-cd}
\usepackage{bm}

% Define a new style with italicized title
\newtheoremstyle{defstyle} % name
    {12pt}        % Space above
    {12pt}        % Space below
    {\upshape}  % Body font
    {}           % Indent amount
    {}          % Theorem head font
    {.}         % Punctuation after theorem head
    {.5em}      % Space after theorem head
    {\textbf{\thmname{#1}\thmnumber{ #2}} (\textit{\thmnote{#3}})}  % Theorem head spec

\newtheoremstyle{remarkstyle} % name
    {12pt}        % Space above
    {12pt}        % Space below
    {\upshape}  % Body font
    {}           % Indent amount
    {}          % Theorem head font
    {.}         % Punctuation after theorem head
    {.5em}      % Space after theorem head
    {\textbf{\thmname{#1}\thmnumber{ #2}}}  % Theorem head spec


\usepackage{amsthm}
\theoremstyle{defstyle}
\newtheorem{theorem}{Theorem}
\newtheorem{definition}{Definition}
\newtheorem{example}{Example}

\theoremstyle{remarkstyle}
\newtheorem{lemma}[theorem]{Lemma}
\newtheorem{proposition}[theorem]{Proposition}
\newtheorem{remark}{Remark}
\newtheorem{corollary}{Corollary}

\numberwithin{equation}{section}
\numberwithin{corollary}{section}
\numberwithin{definition}{section}
\numberwithin{theorem}{section}
\numberwithin{remark}{section}
\numberwithin{example}{section}

% Tocloft package for table of contents customization
\usepackage{tocloft}
\setcounter{tocdepth}{2} % Set depth to show only sections in table of contents

\usepackage{algorithm, algpseudocode, subfig}
\usepackage{graphicx}% Include figure files
\usepackage{dcolumn}% Align table columns on decimal point
\usepackage{bm}% bold math


% Custom command definitions
\newcommand{\ra}{\rangle}
\newcommand{\la}{\langle}
\newcommand{\df}{\dfrac}
\newcommand{\hb}{\hbar}
\newcommand{\intf}{\int_{-\infty}^\infty}
\newcommand{\ang}[1]{\la #1 \ra} % \lang{H} = <H>
\newcommand{\pd}[1]{\partial_{#1}} % Shorthand for partial derivatives
\newcommand{\R}{\mathbb R}
\newcommand{\EL}{\mathbb L}
\newcommand{\La}{\mathcal L}
\newcommand{\D}{\mathcal D}
\newcommand{\p}{\partial}
\newcommand{\mrm}{\mathrm}
\newcommand{\C}{\mathbb C}
\newcommand{\Ha}{\mathbb H}
\newcommand{\Her}{\mathcal H}
\newcommand{\mbf}{\mathbf}
\newcommand{\mca}{\mathcal}
\newcommand{\prf}{\textit{Proof:\,\,\,}}
\newcommand{\Norm}{\mca N}
\newcommand{\argmin}{\mrm{argmin}\,}
\newcommand{\mubold}{\bm{\mu}}
\newcommand{\mbb}{\mathbb}

% Labeled, aligned, equation
\newcommand{\leqalign}[2]{
\begin{equation}\begin{aligned}\label{#1}
#2
\end{aligned}\end{equation}}
\newcommand{\malign}[1]{
\[\begin{aligned}
#1
\end{aligned}\] 
}


\begin{document}

\title{Physics 151 Pset Suggestions}
\date{\today}
\maketitle

\pagebreak
\documentclass[12pt]{article}

% Set margins
\usepackage[margin=0.5in]{geometry}
% Remove paragraph indentation
\setlength{\parindent}{1\baselineskip}
\setlength{\parskip}{0.5\baselineskip}

\usepackage{amsmath}
\usepackage{amssymb}
\usepackage{amsfonts}
\usepackage{hyperref}
\usepackage{amsthm}
\usepackage{tikz-cd}
\usepackage{tikz}
\usepackage{cancel}
\usepackage{mdframed}
\usepackage{enumitem}
\usepackage{tikz-cd}
\usepackage{bm}

\begin{document}

\title{Physics 151 Pset Suggestions}
\date{\today}
\maketitle

\section{Pset 1}

Recall the Lagrangian vector from class (compact form and in components)
\begin{aligned}
    \mathbb L 
    &= \dfrac d {dt} \nabla_{\dot q} \mathcal L - \nabla_q \mathcal L \\ 
    \mathbb L_j 
    &= \dfrac d {dt} \partial_{\dot q_j} \mathcal L - \nabla_q \mathcal L 
\end{aligned}
% Also recall that the Jacobian matrix $J_{a\to b}$ from $a\in \R^m$ to $b\in \R^n$ is 
% \[ 
%     (J_{a\to b})_{ij} = \begin{bmatrix}
%         \nabla_a b_1 & \cdots & \nabla_a b_n 
%     \end{bmatrix} = \dfrac{\partial b_j}{\partial a_i}
% \] 
% so that the chain rule can be written as (compactly and in components) 
% \begin{aligned}
%     \nabla_a f(b) &= J_{a\to b} \nabla_b f(b)\\ 
%     \dfrac{\partial}{\partial a_j} f(b(a_1, \cdots, a_m)) &= 
%     \sum_k \dfrac{\partial f(b)}{\partial b_k} \dfrac{\partial b_k}{\partial a_j}
%     = \sum_k (J_{a\to b})_{jk} (\nabla_b f(b))_k 
% \end{aligned}
% This question asks you to show that the Lagrangian 
% vector transforms covariantly:
% \begin{enumerate}[label=(\alph*), topsep=0pt]
%     \item Given a coordinate transform $x\to q$ so that $q_j=q_j(x_1, \cdots, x_n)$, 
%     it determines a transform $(x, \dot x)\to (q, \dot q)$ of the derivatives as well. 
%     Show that $J_{\dot x\to \dot q} = J_{x\to q}$. 
    
%     \textit{Hint: the transformation $x\to q$ is independent of $\dot x$, 
%     so one can show that $\dot x, \dot q$ are related by a linear transform 
%     $\dot q = J_{x\to q}\dot x$ or, in components, 
%     $\dot q_j = \sum_k (J_{x\to q})_{jk} \dot x_k$}. 

%     \item Show that $J_{x\to \dot q} = \dfrac d {dt} J_{x\to q}$. 
%     \item Explain why $J_{\dot x\to q}=0$. 
%     \item The previous parts give us the full Jacobian of $(x, \dot x)\to (q, \dot q)$ 
%     \[  
%         J_{(x, \dot x)\to (q, \dot q)} = \begin{bmatrix}
%             J_{x\to q} & J_{x\to \dot q} \\ 
%             J_{\dot x\to q} & J_{\dot x\to \dot q} 
%         \end{bmatrix} = \begin{bmatrix}
%             J_{x\to q} & \dfrac d {dt} J_{x\to q} \\ 0 & J_{x\to q}
%         \end{bmatrix}
%     \] 
%     Next show that $\nabla_{\dot x} \mathbb L = J_{\dot x\to \dot q}\nabla_{\dot q}$ 
%     \item Show that 
%     $\nabla_x \mathbb L = \left(J_{x\to q}\nabla_q + J_{x\to \dot q}\nabla_{\dot q}\right)\mathbb L$ 
%     \item Covariance follows easily: show that $\mathbb L_x = J_{x\to q}\mathbb L_q$. 
% \end{enumerate}


% \subsection{Solutions}
% \begin{enumerate}[label=(\alph*), topsep=0pt]
%     \item By the chain rule
%     \[ 
%         \dot q_j = \dfrac d {dt} q_j(x) 
%         = \sum_k \dfrac{\partial q_j}{\partial x_k} \dfrac d {dt} x_k = (J_{x\to q})_{kj}\dot x_k 
%     \] 
% \end{enumerate}

\end{document}
\end{document}
